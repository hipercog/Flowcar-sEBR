% !TEX root = main.tex

\subsection*{Participants}
A convenience sample (N=9, 6 males, 3 females) was recruited via student mailing lists at the University of Helsinki. The participants were between 22-38 years of age (mean 27, SD 3) with normal or corrected-to-normal visual acuity and no history of neurological or psychiatric disease.

All participants were naive about the specific hypotheses and purpose of the study; at the time of recruiting they were informed that the experiment was about game experience and learning. Participants were remunerated with 55 euro worth of cultural vouchers.

Participants were briefed and provided written informed consent before entering the study, and were aware of their legal rights. The study followed guidelines of the Declaration of Helsinki and was approved by the University of Helsinki Ethical review board in humanities and social and behavioural sciences (statement 31/2017; study title MulSimCoLab).

\subsection*{Design}
The experiment was divided into eight sessions on different days over a period of 2-3 weeks. In each session, the participant played five trials of the driving game, each trial lasting 2-4 min depending on their performance, for approximately 15 min of driving time per session.

After each trial, the participant was shown the trial duration and the number of collisions, after which they filled in a self-report questionnaire (FSS). In sessions 1 and 5-8 (lasting approx. an hour), physiological signals were measured in a 5 minutes baseline recording before playing, and during gameplay. In sessions 2-4 (lasting 20 to 30 minutes), no physiological measurements were taken.

\subsection*{Materials}
\paragraph{Game} The experimental task was a custom-made high-speed steering game {\it CogCarSim} designed specifically for the study of Flow and coded in Python. Participants were instructed to avoid as many obstacles they could in order to complete the trial as fast as possible. The game code as used herein is permanently available under open source licence at \url{https://doi.org/10.6084/m9.figshare.7269467}.

\paragraph{Equipment} The game was run on a Corsair Anne Bonny with Intel i7 7700k processor and an Nvidia GTX 1080 graphics card, running Windows 10. Eye-tracking and physiological signals were collected and stored on an Asus UX303L laptop with Debian GNU/Linux 9 OS.

The participant was seated aligned with the mid point of the 55" display screen (LG 55UF85). The screen resolution was 1920 x 1080 pixels, the frame rate was 60 and the refresh rate 60 Hz. The viewing distance was adjusted for each participant (so that they could place their hands on the steering wheel comfortably) and was approximately between 90 and 120 cm from the eye to the screen.

We used Pupil Labs Binocular 120 Hz eye tracker (Pupil Labs UG haftungsbeschränkt, Berlin, Germany), stabilised with a custom-built headband. Pupil Capture software was used to collect the data from the pupil hardware. Gaze direction was calibrated using ten markers on the display, a minimum of three times during the session. Additional calibrations were done if needed. Eye movement signal was recorded at 60 Hz.

Electrodermal activity (EDA, not reported here),  was recorded from the left foot (to minimise motor artefacts). Blood volume pulse (BVP) was measured using a pulse oximeter sensor attached to the left index toe of each participant.

\paragraph{Flow Short Scale} To measure self-reported Flow, participants filled the Flow Short Scale (FSS) after each trial \cite{Rheinberg2003,Engeser2008}. FSS has 10 core items which load the subfactors {\it fluency of performance} (6 items) and {\it absorption by activity} (4 items); plus 3 items for {\it perceived importance}. The response format of FSS is a 7-point Likert scale ranging from {\it Not at all} to {\it Very much}. Higher scores on the scales indicate higher experienced Flow and perceived importance.

The Flow score used herein was formed by averaging items in the {\it fluency of performance} and {\it absorption by activity} subfactors; {\it perceived importance} was used separately in analyses.

% (+3 additional items)
In addition to the 13 main items asked after every trial, participants were asked at the end of every session to report 3 more items measuring the fit of skills and demands of the task \cite{Rheinberg2003}.

\subsection*{Procedure}
In their first session, participants were informed about the procedure of the study and asked to fill in a background information questionnaire, including information on health, driving experience and gaming experience, and an informed consent form.

The sessions were managed by two research assistants at a time, who observed the measurement out of participants' line of sight. Before each session participants filled in a session-wise questionnaire on the use of contact lenses, restedness, and medication, caffeine, and nicotine intake.

In sessions 2 to 4, participants played five trials straight after filling in the session-wise questionnaire. The FSS was filled after each trial. In sessions with physiological measurements (1 and 5 to 8), participants were dressed in physiological sensors and an eye-tracking headset, seated in the driving seat in quiet, low-light conditions for baseline measurement. They were asked to sit still for five minutes, looking at a dark blue screen, while baseline was recorded. After baseline recording, participants played five game trials, filling FSS after each trial.

\subsection*{Signal preprocessing and analysis}
Eye blinks were counted manually from the eye tracking videos recorded during baseline period of sessions 1 and 5--8. Three-minute periods were considered sufficient for this purpose, thus the first and last minute from each five-minute recording were omitted to obtain the most stable period of baseline. Four measurements (out of 40) were excluded due to measurement problems.

All fast and simultaneous movements of both eyelids were counted as blinks (even if the eyelid did not fully close). To ensure reliable blink identification, two of the authors independently counted the number of blinks in sessions 1, 6 and 8, and inter-rater reliability of the counts was calculated as 98.7\% (see below). We considered this high enough to have blinks in remaining sessions 5 and 7 counted by only one experimenter.

We calculated the level of consensus between the two raters as follows. Separately for each participant and session, we divide the difference of two raters' blink counts by the mean of those counts, and then subtract the quotient from 1 to obtain a percentage. All percentages are then averaged to give the overall measure of inter-rater reliability. The session-wise reliability scores also had low variability (mean of standard deviations = 0.01).

The final spontaneous eye blink rate was calculated as median blinks per minute during the baseline measurement sessions.

\subsection*{Statistical methods}
All statistical data processing reported herein was implemented with {\sf R} platform for statistical computing \cite{R2014}. Where possible, exact corrected {\it p}-values are reported; inequalities are reported where exact values were not available. All {\it p}-values were corrected for multiple comparisons using Bonferroni-Holm. For all simple correlations we calculated Pearson's correlation coefficient, because all data in these tests were shown to be normally distributed by Shapiro-wilk tests and associated Q-Q plots. The {\sf R} code and data used to produce all analyses and figures is permanently available online at % FIXME - NEW FIGSHARE FOR BLINK PAPER
% tests   pvals    padj
% 1           sEBRxLC 0.21480 1.00000
% 2         sEBRxFlow 0.70000 1.00000
% 3      FlowXsEBRdev 0.00300 0.04500
% 4  sEBRxLCxFlow+1SD 0.00001 0.00018
% 5 sEBRxLCxFlow_mean 0.00001 0.00018
% 6  sEBRxLCxFlow-1SD 0.15000 1.00000


\subsubsection*{Interaction analysis}
For {\sf RQ3}, to conduct a {\it simple slopes} interaction analysis we used the {\sf jtools} package for {\sf R} \cite{jtools}. The simple slopes analysis was conducted on a linear regression model (using {\it lm} function in {\sf R}) with sEBR as dependent variable; predictors were main effects of, and interaction of, Flow and LC. The slope of LC was estimated when Flow was held constant at its mean$\pm$1SD, producing the estimates shown in Table~\ref{tab:simpslopes} (see Results).


%%%%%%%%%%%%%%% END OF METHODS %%%%%%%%%%%%%%%
