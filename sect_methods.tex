% !TEX root = main.tex

\paragraph{Overview:}
Participants learned to play a custom-made high-speed steering game (for game video see \url{https://doi.org/10.6084/m9.figshare.7269395.v1}). The aim of the game was to steer a blue cube through a course with randomly placed red obstacles at the highest possible speed, so performance was measured by duration of the trial. Collision with obstacles reduced speed by a fixed amount. The cube started each game at a fixed forward velocity, which increased at a constant rate. The lateral position of the cube was controlled by steering wheel. The game was specifically designed to elicit Flow by constantly balancing task demand with participant skill, and providing clear immediate feedback.

Participants played the game for forty trials across eight sessions, over a period of 2-3 weeks, which was sufficient to achieve good proficiency in this task with no ceiling effect. The 10 item Flow Short Scale \cite{Engeser2008} was filled after each trial to probe self-reported Flow in the task. Physiological data were recorded, during task and five minutes of baseline, in sessions one and five-to-eight.


\paragraph{Participants}
A convenience sample (N=9, 6 males, 3 females) was recruited, between 22-38 years of age (mean 27, SD 3), with normal or corrected-to-normal visual acuity and no history of neurological or psychiatric disease.

All participants were naive about the specific hypotheses and purpose of the study; at the time of recruiting they were informed that the experiment was about game experience and learning. Participants were remunerated with 55 euro worth of cultural vouchers.

Participants were briefed and gave written informed consent before the study, and remunerated after. The study followed guidelines of the Declaration of Helsinki and was approved by the University of Helsinki Ethical review board in humanities and social and behavioural sciences (statement 31/2017; study title MulSimCoLab).

\subsection*{Materials}
\paragraph{Task \& Equipment} Participants played the custom high-speed steering game {\it CogCarSim} (game code permanently available under open source licence at \url{https://doi.org/10.6084/m9.figshare.7269467}). Separate computers were used to run the game and record the physiology data.

Participants were seated aligned with the mid point of the 55" display screen (LG 55UF85). The screen resolution was 1920 x 1080 pixels, the frame rate was 60 and the refresh rate 60 Hz. The viewing distance was 90--120 cm from the eye to the screen.

Sessions 1,5,6,7,8 had physiological measurements. Participants were dressed in physiological sensors and an eye-tracking headset, seated in the driving seat in quiet, low-light conditions for baseline measurement, where they sat still for five minutes looking at a dark blue screen.

We used Pupil Labs Binocular 120 Hz eye tracker (Pupil Labs UG haftungsbeschränkt, Berlin, Germany), stabilised with a custom-built headband, with data acquisition via Pupil Capture software. Gaze direction was calibrated using ten markers on the display, three times per session.

Electrodermal activity (EDA, not reported here),  was recorded from the left foot (to minimise motor artefacts). Blood volume pulse (BVP) was measured using a pulse oximeter sensor attached to the left index toe.

\paragraph{Flow Short Scale} To measure self-reported Flow, participants filled the Flow Short Scale (FSS) after each trial \cite{Rheinberg2003,Engeser2008}. FSS has 10 core items which load the subfactors {\it fluency of performance} (6 items) and {\it absorption by activity} (4 items); plus 3 items for {\it perceived importance}. The response format of FSS is a 7-point Likert scale ranging from {\it Not at all} to {\it Very much}. Higher scores on the scales indicate higher experienced Flow and perceived importance.

The Flow score used herein was formed by averaging items in the {\it fluency of performance} and {\it absorption by activity} subfactors; {\it perceived importance} was used separately in analyses.

% (+3 additional items)
In addition to the 13 main items asked after every trial, participants were asked at the end of every session to report 3 more items measuring the fit of skills and demands of the task \cite{Rheinberg2003}.

% \subsection*{Procedure}
% In their first session, participants were informed about the procedure of the study and asked to fill in a background information questionnaire, including information on health, driving experience and gaming experience, and an informed consent form.
%
% The sessions were managed by two research assistants at a time, who observed the measurement out of participants' line of sight. Before each session participants filled in a session-wise questionnaire on the use of contact lenses, restedness, and medication, caffeine, and nicotine intake.
%
% In sessions 2 to 4, participants played five trials straight after filling in the session-wise questionnaire. The FSS was filled after each trial. In sessions with physiological measurements (1 and 5 to 8), participants were dressed in physiological sensors and an eye-tracking headset, seated in the driving seat in quiet, low-light conditions for baseline measurement. They were asked to sit still for five minutes, looking at a dark blue screen, while baseline was recorded. After baseline recording, participants played five game trials, filling FSS after each trial.

\subsection*{Signal preprocessing and analysis}
Eye blinks were counted manually from the middle three minutes of each baseline eye-tracking video (from sessions 1 \& 5--8). Four measurements (out of 40) were excluded due to measurement problems.

All fast and simultaneous movements of both eyelids were counted as blinks (even if the eyelid did not fully close). To ensure reliable blink identification, two of the authors independently counted the number of blinks in sessions 1, 6 and 8, and inter-rater reliability of the counts was calculated as 98.7\% (see below). We considered this high enough to have blinks in remaining sessions 5 and 7 counted by only one experimenter.

We calculated the level of consensus between the two raters as follows. Separately for each participant and session, we divide the difference of two raters' blink counts by the mean of those counts, and then subtract the quotient from 1 to obtain a percentage, which were then averaged to give total inter-rater reliability. The session-wise reliability scores also had low variability (mean of standard deviations = 0.01).

The final spontaneous eye blink rate was calculated as median blinks per minute during the baseline measurement sessions.

\subsection*{Statistical methods}
All statistical data processing reported herein was implemented with {\sf R} platform for statistical computing \cite{R2014}. {\sf R} code and data used to produce all analyses and figures is permanently available online at % FIXME - NEW FIGSHARE FOR BLINK PAPER

Where possible, exact corrected {\it p}-values are reported; inequalities are reported where exact values were not available.
All {\it p}-values were corrected for multiple comparisons using Bonferroni-Holm.
% For all simple correlations we calculated Pearson's correlation coefficient, because all data in these tests were shown to be normally distributed by Shapiro-wilk tests and associated Q-Q plots.
% tests   pvals    padj
% 1           sEBRxLC 0.21480 1.00000
% 2         sEBRxFlow 0.70000 1.00000
% 3      FlowXsEBRdev 0.00300 0.04500
% 4  sEBRxLCxFlow+1SD 0.00001 0.00018
% 5 sEBRxLCxFlow_mean 0.00001 0.00018
% 6  sEBRxLCxFlow-1SD 0.15000 1.00000


\paragraph{Interaction analysis:}
To conduct all {\it simple slopes} interaction analyses we used the {\sf jtools} package for {\sf R} \cite{jtools}. The {\bf RQ1} simple slopes analysis was conducted on a linear regression model (using {\it lm} function in {\sf R}) with sEBR as dependent variable; predictors were main effects of, and interaction of, Flow and LC. The slope of LC was estimated when Flow was held constant at its mean$\pm$1SD, producing the estimates shown in Table~\ref{tab:simpslopes} (see Results).

\paragraph{Bayesian analysis:}

%%%%%%%%%%%%%%% END OF METHODS %%%%%%%%%%%%%%%
